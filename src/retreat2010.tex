\documentclass[portrait,final]{baposter}
%\documentclass[a4shrink,portrait,final]{baposter}
% Usa a4shrink for an a4 sized paper.

\tracingstats=2

\usepackage{times}
\usepackage{calc}
\usepackage{graphicx}
\usepackage{amsmath}
\usepackage{amssymb}
\usepackage{relsize}
\usepackage{multirow}
\usepackage{bm}

%\usepackage{gnuplot-lua-tikz}

\usepackage{graphicx}
\usepackage{multicol}

\usepackage{pgfbaselayers}
\pgfdeclarelayer{background}
\pgfdeclarelayer{foreground}
\pgfsetlayers{background,main,foreground}

\usepackage{helvet}
%\usepackage{bookman}
\usepackage{palatino}

\newcommand{\captionfont}{\footnotesize}

\selectcolormodel{cmyk}

\graphicspath{{../img/}}

% Multicol Settings
\setlength{\columnsep}{0.7em}
\setlength{\columnseprule}{0mm}


% Save space in lists. Use this after the opening of the list
\newcommand{\compresslist}{%
\setlength{\itemsep}{1pt}%
\setlength{\parskip}{0pt}%
\setlength{\parsep}{0pt}%
}

\definecolor{silver}{cmyk}{0,0,0,0.3}
\definecolor{yellow}{cmyk}{0,0,0.9,0.0}
\definecolor{reddishyellow}{cmyk}{0,0.22,1.0,0.0}
\definecolor{black}{cmyk}{0,0,0.0,1.0}
\definecolor{darkYellow}{cmyk}{0,0,1.0,0.5}
\definecolor{darkSilver}{cmyk}{0,0,0,0.1}

\definecolor{lightyellow}{cmyk}{0,0,0.3,0.0}
\definecolor{lighteryellow}{cmyk}{0,0,0.1,0.0}
\definecolor{lighteryellow}{cmyk}{0,0,0.1,0.0}
\definecolor{lightestyellow}{cmyk}{0,0,0.05,0.0}


\begin{document}
\begin{poster}{%
    grid=no,
    colspacing=1em,
    bgColorOne=lighteryellow,
    bgColorTwo=lightestyellow,
    borderColor=reddishyellow,
    headerColorOne=yellow,
    headerColorTwo=reddishyellow,
    headerFontColor=black,
    boxColorOne=white, %lightyellow,
    boxColorTwo=lighteryellow,
    textborder=roundedleft,
    eyecatcher=no,
    headerborder=open,
    headerheight=0.08\textheight,
    headershape=roundedright,
    headershade=plain,
    headerfont=\Large\textsf, %Sans Serif
    boxshade=plain,
    background=plain,
    linewidth=2pt}
  {eyecatch} 
  {Getting more with less: sectioning methods and \\ angular
    control in a widefield microscope} 
  {Martin Kielhorn, Daniel Appelt, Susan Cox, Rainer Heintzmann} {
    \includegraphics[height=5em]{logo}
  }
  \headerbox{Improving sectioning with the apotome}
  {name=daniel,column=0,span=3,row=0}{
    {}
\begin{multicols}{3}
    {\bf Introduction}
    One method to obtain optical sectioned data is ``structured
    illumination microscopy'' (SIM), as implemented in commercial
    devices like the Zeiss ApoTome or the Qioptiq Optigrid. In this
    procedure, a grating is inserted in the illumination path of a
    conventional fluorescence microscope and projected into sample
    space (Fig.1). Taking three images with three different grating
    positions (Fig.2) allows one to computationally remove
    out-of-focus light. As a result, contrast and resolution in
    z-direction are improved (Fig.3).

    {\bf The principle} Fig.4 shows the effect of the grating on the
    detected fluorophore intensity. Depending on their z-position the
    moving of the grating will lead to an intensity modulation. This
    fact is essential to computationally remove out-of-focus light.

    {\bf SIM reconstruction} Different algorithms can be used to
    remove out-of-focus light. A very simple one is explained in
    Fig.5.

    \begin{minipage}[h]{7.5cm}
      \begin{center}
      \includegraphics[width=4cm]{../../daniel/sketch.png}
      \end{center}
      {\smaller Figure 1. The grating is inserted at the field
        diaphragm in the illumination path of a conventional
        fluorescence microscope and projected onto the sample using an
        incoherent light source.}
    \end{minipage}
    
    \begin{minipage}[h]{7.5cm}
      \includegraphics[width=7.5cm]{../../daniel/grating-img.png}
      {\smaller Figure 2. Shown are the three raw images, each with a
        different grating position. The same grid line is marked with
        a red bar in each image. After the first image is taken, the
        grating is moved linearly by $1/3$ of the grating period length
        to capture the second image, and another $1/3$ to capture the
        third image. Specimen: Adult rat cardiomyocytes, stained with
        monoclonal mouse antibodies against the titin epitope T12 and
        secondary Cy2-conjugated anti mouse immunoglobulin
        antibodies.}
    \end{minipage}
    \begin{minipage}[h]{7.5cm}
      \medskip
      \begin{center}
        \includegraphics[width=6cm]{../../daniel/reconstruct_2.png}
      \end{center}
      {\smaller Figure 3. Shown are two images of adult rat
        cardiomyocytes stained with monoclonal mouse antibodies against
        the titin epitope T12 and secondary Cy2-conjugated anti mouse
        immunoglobulin antibodies. Left: widefield, Right: SIM
        reconstruction.}
    \end{minipage}
    
    \begin{minipage}[h]{7.5cm}
      \vspace{-3mm}
      \begin{center}
        \includegraphics[width=7.5cm]{../../daniel/explain.png}
      \end{center}
      {\smaller Figure 4. Simulation of a x-z cross section through
        sample space showing the three raw images taken at different
        grating positions and the effect on the detected intensity of
        the fluorophores I and II. The moving of the grating leads to an
        intensity modulation for fluorophore I in but not for
        fluorophore II.}
    \end{minipage}
    \begin{minipage}[h]{7.5cm}
        \medskip
        \begin{center}
          \includegraphics[width=5cm]{../../daniel/screen-table.png}
        \end{center}
        {\smaller Figure 5. Max-Min - Algorithm for the situation shown in
          Fig. 4. Every pixel value consists of modulated in-focus
          information (II1, II2, II3) and unmodulated out-of-focus
          information (III). The lowest of the three values that were
          taken for each pixel Imin is substracted from the highest of the
          three pixel values Imax. The final image I contains exclusively
          in-focus information. The unmodulated out-of-focus information
          III has been removed.}
      \end{minipage}
    \end{multicols}
  }
  \headerbox{Sectioning methods}
  {name=susan,column=0,span=3,below=daniel}{
    {}
\begin{multicols}{3}
   Distinguish between in-focus and out of
   focus light using a grating projected onto the sample with a spatial
   light modulator in the image plane.
  
   \begin{minipage}[h]{7.5cm}
     \medskip
     \begin{center}
       \includegraphics[width=7.5cm]{../../susan/section.png}
     \end{center}
     {\smaller Figure 6. }
   \end{minipage}
   
   1 Apotome    3 x
 
   2 HiLo    

   3 Adapted HiLo   2x

   Adapted HiLo method has smallest light dose to create a sectioned
   image and was therefore selected.

   Controlled light exposure
   1 Initial short exposure images taken, sectioned image reconstructed
   2 Create mask to block very bright, very dark and out-of-focus areas.
   3 Take long exposure images with grating and mask
   4 Reconstruct final sectioned image using all  information
  
   Leads to factor of 9 improvement in bleaching of bead sample, with
   same signal-to-noise in both images in medium intensity areas.
\end{multicols}
  }
\headerbox{Spatio-angular illumination microscopy}
{name=ich,column=0,span=3,below=susan}{
  {}
\begin{multicols}{3}
  In some biological specimen it is possible to exploit a sparse
  fluorophore concentration. In our modified wide field microscope we
  can shape the light distribution of the excitation light in the
  front and back focal plane. This facilitates excitation of in-focus
  fluorophores without or with much reduced exposure of out-of-focus
  fluorophores. The benefits will be twofold: A smaller out-of-focus
  blur reduces its contribution to the photon noise and therefore
  improves the signal noise ratio in the image. Reducing unnecessary
  exposure also protects the specimen. This can be important when
  observing development of embryos.

  In order to use of the light shaping to our advantage it is
  necessary to account for the fluorophore concentration in the
  specimen. Therefor we represent the specimen (c. elegans embryo) as
  a sufficiently accurate model (each nucleus is represented as a
  sphere).  By raytracing it is then possible to find ``unobstructed''
  paths from the back focal plane through the back focal plane and the
  rest of the embryo. Limiting the illumination to these regions will
  excite one nucleus in the focal plane and protect all the nuclei
  from exposure.


  \includegraphics[width=7.5cm]{../../ich/2a.png}
  \emph{Raytracing model of a c.~elegans embryo.}
 

  The Figure above shows a model of an immersion objective and the
  path of excitation rays from the back focal plane towards the
  sample. The model of the specimen was constructed from a confocal
  stack of a living embryo with H2B-tagged histones. The excitation
  rays fill the biggest possible circle in the back focal plane, so
  that they illuminate a nucleus in the periphery of the embryo
  without crossing the other spheres. In the next Figure the the
  average intersection length of the rays with the out-of-focus
  spheres is plotted. A circular window with relative radius $r$
  (compared to back focal plane radius) is scanned over the back focal
  plane. In black regions the rays don't intersect any out-of-focus
  spheres. In order to find an optimal illumination shape the window
  radius is incremented until the minimum looses contrast.
  
  \includegraphics[width=7.5cm]{../../ich/2b.png}
  \emph{The circular shapes depict an estimate of out-of-focus
    exposure for different window sizes in the back focal plane. These
    values were obtained by tracing rays through the sphere model of
    the embryo.}
  
\end{multicols}
  }
  \headerbox{Literature}{name=ref,column=0,span=1,above=bottom}{
    {} 
    \smaller
    \vspace{-0.4em}
     \bibliographystyle{ieee}
     \renewcommand{\section}[2]{\vskip 0.05em}
     \begin{thebibliography}{1}\itemsep=-0.01em
       \setlength{\baselineskip}{0.4em}
     \bibitem{2004huisken}
       J.~Huisken, et al.. 
       \newblock Optical sectioning deep inside living embryos by
       selective plane illumination microscopy
       \newblock In {\em Science, 305, 1007-1009 (2004)}
     \bibitem{2007hoebe}
       R.~A.~Hoebe, et al..
       \newblock Controlled light-exposure microscopy reduces
       photobleaching and phototoxicity in fluorescence live-cell
       imaging
       \newblock In {\em Nat Biotech, 25, 249-253 (2007)}
     \end{thebibliography}}
  \headerbox{Acknowledgements}{name=ack,column=1,span=2,above=bottom}{
    {} \smaller     
    We thank Spencer Shorte, Christophe Machu, Jean-Yves Tinevez
    (Institut Pasteur, Paris, France), Vincent Galy (Uni Pierre et
    Marie Curie, Paris, France), Erhard Ipp (In-Vision, Vienna,
    Austria), Joel Seligson (KLA Tencor, Migdal Ha'emek, Israel) and
    Joerg Heber (Fraunhofer IPMS, Dresden, Germany) for helpful
    discussions.

    S. Cox and M. Kielhorn are funded by the EU FP7. D. Appelt is
    funded by the MRC.}
\end{poster}%
\end{document}
